\newcommand{\beq}{\begin{equation}}
\newcommand{\eeq}{\end{equation}}


\def\ie{{\it i.e.}}
\def\eg{{\it e.g.}}
\newcommand{\p}{\partial}
\newcommand{\mc}[1]{\mathcal{#1}}
\newcommand{\md}{\mathcal{D}}
\newcommand{\wt}{\widetilde}
\newcommand{\ov}{\overline}
\newcommand{\suc}{{\rm SU}_{\rm C}(3)}
\newcommand{\sul}{{\rm SU}_{\rm L}(2)}
\newcommand{\ue}{{\rm U}(1)}
\newcommand{\GeV}{{\rm GeV}}
\newcommand{\eV}{{\rm eV}}
\newcommand{\ha}{\frac{1}{2}}
%\newcommand{\su3}{{\rm SU}_{\rm C}(3)}
%%%%%%%%%%%%%%%%%%%%%%%%%%%%%%%%%%%%%%%
%  Slash character...
\def\slashed#1{\setbox0=\hbox{$#1$}             % set a box for #1
   \dimen0=\wd0                                 % and get its size
   \setbox1=\hbox{/} \dimen1=\wd1               % get size of /
   \ifdim\dimen0>\dimen1                        % #1 is bigger
      \rlap{\hbox to \dimen0{\hfil/\hfil}}      % so center / in box
      #1                                        % and print #1
   \else                                        % / is bigger
      \rlap{\hbox to \dimen1{\hfil$#1$\hfil}}   % so center #1
      /                                         % and print /
   \fi}                                        %
%%EXAMPLE:  $\slashed{E}$ or $\slashed{E}_{t}$

\newcommand{\LUV}{\Lambda_{\rm UV}}


\newcommand{\lgr}{\left\lgroup}
\newcommand{\rgr}{\right\rgroup}

